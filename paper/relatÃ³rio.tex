\documentclass{article}
\usepackage[utf8]{inputenc}
\usepackage[portuges]{babel}
\usepackage{graphicx}
\usepackage[labelformat=empty]{caption}
\usepackage{color}
\usepackage[dvipsnames]{xcolor}
\usepackage{float}
\usepackage{textcomp}
\usepackage{listings}
\usepackage{rotating}

\definecolor{solarized@base03}{HTML}{002B36}
\definecolor{solarized@base02}{HTML}{073642}
\definecolor{solarized@base01}{HTML}{586e75}
\definecolor{solarized@base00}{HTML}{657b83}
\definecolor{solarized@base0}{HTML}{839496}
\definecolor{solarized@base1}{HTML}{93a1a1}
\definecolor{solarized@base2}{HTML}{EEE8D5}
\definecolor{solarized@base3}{HTML}{FDF6E3}
\definecolor{solarized@yellow}{HTML}{B58900}
\definecolor{solarized@orange}{HTML}{CB4B16}
\definecolor{solarized@red}{HTML}{DC322F}
\definecolor{solarized@magenta}{HTML}{D33682}
\definecolor{solarized@violet}{HTML}{6C71C4}
\definecolor{solarized@blue}{HTML}{268BD2}
\definecolor{solarized@cyan}{HTML}{2AA198}
\definecolor{solarized@green}{HTML}{859900}

\lstset{
  language=C,
  upquote=true,
  columns=fixed,
  tabsize=2,
  extendedchars=true,
  breaklines=true,
  numbers=left,
  numbersep=5pt,
  rulesepcolor=\color{solarized@base03},
  numberstyle=\tiny\color{solarized@base01},
  basicstyle=\footnotesize\ttfamily,
  keywordstyle=\color{solarized@green},
  stringstyle=\color{solarized@yellow}\ttfamily,
  identifierstyle=\color{solarized@blue},
  commentstyle=\color{solarized@base01},
  emphstyle=\color{solarized@red}
}

\graphicspath{ {images/} }

\title{\bf{\textcolor{Mahogany}{Relatório do Projeto de \emph{SO}}}}
\author{António Sérgio (a78296) \and Hugo Brandão (a78582) \and Tiago Alves (a78218) }
\date{Abril 2017}

\begin{document}
\maketitle
\begin{center}

\par
\large {Grupo 2}
\par
\large {Universidade do Minho}
\par
\large {Departamento de Informática}
\par
\large {Sistemas Operativos}
\vfill
\bf Mestrado Integrado em Engenharia Informática
\par
\includegraphics[width=5cm]{UM}
\end{center}

\clearpage

\tableofcontents

\clearpage
\section{Introdução}
\par Este trabalho irá ser realizado no âmbito da unidade curricular \emph{Sistemas Operativos}, no qual pretendemos construir um sistema de \emph{stream processing}. Basicamente teremos que criar uma variedade de componentes que irão receber eventos e irão filtrá-los, modificá-los, entre outras funcionalidades.

\par Para conseguir implementar este tema iremos utilizar várias funcionalidades em \textbf{C} que fomos aprendendo ao longo do semestre nas aulas desta unidade curricular, tais como \emph{pipes}, \emph{fork}, \emph{exec}, \emph{read}, etc.

\par Após a leitura do enunciado do trabalho prático pensámos que as primeiras funções são relativamente acessíveis, todavia as funções do controlador como \emph{connect}, \emph{disconnect}, \emph{node} e \emph{inject}, causarão mais problemas visto que já requerem um maior cuidado na criação de processos e na abertura e fecho dos pipes. Contudo estamos motivados para realizar este trabalho.

\par Neste relatório iremos relatar todos os métodos utilizados para o bom funcionamento de cada função, os problemas que fomos encontrando e como os combatemos. Estamos à espera de uma jornada longa e com muitos contratempos, mas esperamos concretizar todos os objetivos por nós traçados.

\clearpage

\section{Componentes}
\par Esta parte do trabalho prático foi rapidamente concluída. Tivemos alguns contratempos mas céleremente superados, sem dúvida que esta foi a parte mais simples do trabalho, visto que apenas envolveu funcionalidades muito básicas que aprendemos nesta \textbf{UC}. Todas as implementações foram feitas como programas separados, isto é, leem do \emph{stdin} e escrevem no \emph{stdout}.
\subsection{Const}
\par Esta função simplesmente acrescenta um parametro passado como argumento no fim de cada linha lida, ou seja, quando fazemos \emph{./const 10} todas as linhas a seguir escritas vão ser reescritras com um \emph{:10} no fim.
\par Por exemplo:
\begin{lstlisting}
./const 10
a:1:2
a:1:2:10
bind:2:slot:ok
bind:2:slot:ok:10
\end{lstlisting}
Esta função foi fácil de implementar, pois só utilizamos funcionalidades como o \emph{read} e o \emph{write}.
\begin{lstlisting}
while((i = readln(0, buf, PIPE_BUF))>0){
	buf[i-1] = ':';
	strcat(buf, argv[1]);
    strcat(buf,"\n");
	i+=strlen(argv[1]);
	write(1, buf, i+1);
	memset(buf, 0, i);
}
\end{lstlisting}
\subsection{Filter}
A filter recebe como argumentos dois números de colunas e uma operação condicional, e vai aplicar essa operação às duas colunas, caso seja verdade imprime a linha, caso contrário não imprime nada.
\par No exemplo seguinte, caso a segunda coluna seja menor que a quarta, imprimimos, caso contrário, não imprimimos.
\begin{lstlisting}
./filter 2 < 4
a:1:2:2
a:1:2:2
bind:2:slot:2
b:5:t:10
b:5:t:10
\end{lstlisting}
\par Foi uma função foi fácil de implementar. O único problema foi descobrir como separar as colunas, mas após alguma procura descobrimos uma implementação simples e terminamos rapidamente.
\par Aqui podemos ver um excerto de código, onde mostra como as colunas são processadas:
\begin{lstlisting}
void  filter (void* buf, char* copy, int size, char* operation, int column, int column2){
    int c=1;
    int product1 , product2;
    char *num1, *num2, copy2[PIPE_BUF];
    strcpy(copy2, copy);

    for(num1 = strtok(copy, ":"); num1 != NULL && c<column; num1 = strtok(NULL, ":")){
        c++;
    }
    if (num1==NULL){
        perror("Missing columns!");
        exit(-1);
    }
    c=1;
    product1 = atoi(num1);
    for(num2 = strtok(copy2, ":"); num2 != NULL && c<column2; num2 = strtok(NULL, ":")){
        c++;
    }
    if (num2==NULL){
        perror("Missing columns!");
        exit(-1);
    }
    else {
        product2 = atoi(num2);
        operate(buf, size, operation, product1, product2);

    }
}
\end{lstlisting}
\subsection{Window}
\par A window recebe como argumentos um número de uma coluna(c), uma função(f) e um número de linhas(l). Após receber uma linha, esta vai à coluna(c) e guarda-a. De seguida, cálcula a função(f) às últimas (l) linhas.
\par Por exemplo, \emph{./window 3 max 2} vai guardando a terceira coluna de cada linha e faz o \emph{max}(máximo) da terceira coluna das últimas 2 linhas.
\begin{lstlisting}
./window 3 max 2
a:3:3
a:3:3:0 //visto que ainda nao ha colunas guardadas, o maximo e zero
b:5:6:7
b:5:6:7:3 //visto que so escrevemos uma linha, o maximo e essa linha
c:3:5
c:3:5:6 // a partir de agora faz o maximo das ultimas duas linhas, como as colunas guardadas sao "3" e "6", o maximo e 6
\end{lstlisting}
\par Apesar de parecer uma função mais complicada, não foi muito difícil de implementar, visto que já sabiamos como separar as colunas e as operações pedidas eram de fácil implementação.
\par Aqui podemos ver um excerto de código, onde mostramos como a implementação e muito parecida à da filter:
\begin{lstlisting}
void window(void* buf, char* copy, int size, char* operation, int coluna){
    int c=1;
    int product;
    char* num;
    for(num = strtok(copy, ":"); num != NULL && c<coluna; num = strtok(NULL, ":")){
        c++;
    }
    if (num==NULL){
        perror("missing columns");
        exit(-1);
    }
    else {
        product = atoi(num);

        operate(buf, size, operation, product);
    }
}
\end{lstlisting}

\subsection{Spawn}
A spawn recebe como argumento um comando e seus argumentos. De seguida executa o comando com os argumentos passados e guarda o seu exit status. Após isso vai processar linhas e escreve o \emph{exitstatus} no fim da cada linha.
\par Por exemplo, se fizermos \emph{./spawn ls -l}, caso esta operação corra bem acrescenta (exit status = 0):
\begin{lstlisting}
./spawn ls -l
a:1:2:2
a:1:2:2:0
b:5:t:10
b:5:t:10:0
\end{lstlisting}
\par Devido ao exec esta função causou alguns problemas. No entanto, após algum trabalho com forks, a função ficou rapidamente operacional.
\par Aqui podemos ver um excerto de código, onde mostra como o comando é executado e como guardamos o \emph{exit status}.
\begin{lstlisting}
void spawn(char* buf, char* copy, char** argv, int sizeBuf){
    int status;
    char execResult[4];
    pid_t p;
    p = fork();
    if(!p){
        execvp(argv[1], &argv[1]);
        _exit(-1);
    }
    else{
        wait(&status);
        sprintf(execResult, "%d", WEXITSTATUS(status));
        writeOP(buf, execResult, sizeBuf);
    }
}
\end{lstlisting}

\clearpage

\section{Controlador}
\par Esta parte do trabalho foi a mais longa, pois já começou a envolver gestão de processos e pipes, tanto anónimos como com nome. Ocorreram pequenos problemas que nos ocuparam horas para resolver, visto que a maneira como abrimos e fechamos os pipes, e como estes comunicam entre si em processos diferentes tem de ser meticulosamente cuidada.

\subsection{Node}
\par A node recebe um número, e um componente com os seus argumentos. O número(n) recebido serve para identifcar este node. Este vai executar o componente com os argumentos, e passar um input recebido no \emph{fifo de entrada} desse node como input desse componente e, de seguida, guarda o output gerado e vai redirecioná-lo para o \emph{fifo de saída} desse node.
\par Um nodo é composto por 3 processos-filho. Um que lê constantemente do \emph{fifo de entrada} e escreve no pipe anónimo de entrada. Um que redireciona o \emph{stdin} para o pipe anónimo de entrada e o \emph{stdout} para o pipe anónimo de saída através de \emph{dups} e executa o comando dado como argumento através de um exec. O último lê permanentemente do pipe anónimo de saída e escreve o que leu para o \emph{fifo de saída}.
\par A metodologia que utilizamos para a identificação dos pipes usa a concatenação de strings para uma fácil identificação. Por exemplo, o \emph{fifo de entrada} do nodo 1 será o "in1" e o do nodo 77 será o "in77", tal como o \emph{fifo de sáida} do nodo 1 será o "out1" e o do nodo 77 "out77".

\begin{figure}[ht]
\centering
\includegraphics[width=5cm]{nodo.jpg}
\caption{Estrutura de um nodo.}
\end{figure}

\par Por exemplo, quando fazemos \emph{./node 1 ./const 10}, este vai criar um node que é identificado com o número 1, e todo o input recebido irá ser recebido durante execução do componente const, e todo o output irá sair com um \emph{:10} no final.
\begin{lstlisting}
dupexecp = fork();
			if(!dupexecp){
				close(nonamepipein[WRITE]);
				close(nonamepipeout[READ]);
				dup2(nonamepipein[READ], 0);//dup do std in para o pipe in
				dup2(nonamepipeout[WRITE], 1);//dup do std out para o pipe out
				execv(argv[2], &argv[2]);
				execvp(argv[2], &argv[2]); //no caso do comando ser um pre definido o exec anterior falhara, executando entao um execvp
			}
\end{lstlisting}

\par Aqui está um excerto de código utilizado que mostra como conseguimos colocar o input recebido na node, como input para o componente a ser executado. Usámos \emph{fork} porque necessitamos de um processo apenas para o \emph{execv}, visto que este vai substituir o processo onde está inserido, e utilizamos os \emph{dup2} para "substituir" o stdin e o stdout por \emph{pipes anónimos}, que iram fornecer informação ao componente e guardar o seu output.
\par A teoria da implementação foi, razoavelmente, fácil de criar, no entanto a sua implementação foi mais difícil, posto que criámos muitos processos e há muitos pipes anónimos e com nome a comunicarem entre si, logo tivemos de ser muito cuidadosos com o que faziamos em cada processo.

\subsection{Connect}
\par A connect recebe números como argumentos. Estes números representam nodes diferentes, e a connect vai fazer com que estes comuniquem entre si, ou seja, tudo que sair do node passado primeiro como argumento (estiver no \emph{fifo de saída}), vai entrar nos nodes passados a seguir nos argumentos (entrar no \emph{fifo de entrada}).
\par Por exemplo, \emph{./connect 1 2} faz com que tudo que entra que esteja na saída do \emph{node 1} entre no \emph{node 2}.
\begin{lstlisting}
	int id = atoi(argv[1]), i;
	if(nodes[id]==0){
		perror("nodo nao existe");
		return 0;
	}
	if(connects[id]){
		kill(connects[id], SIGTERM);
	}
	for(i=2; i<argc; i++){
		if(nodes[atoi(argv[i])]==0){
			perror("nodo nao existe");
			return 0;
		}
		connections[id][atoi(argv[i])] = 1;
	}
\end{lstlisting}
\par No excerto de código anterior mostra como controlamos todas as possibilidades. Para isso utilizamos dois arrays e uma matriz. Caso o node passado como primeiro argumento, ou seja, onde vamos buscar informação, já tiver conexões com outros nodes, o array \emph{connects} guarda  \emph{pid} desse processo, e matamos esse processo pois iremos fazer um processo que faça todas as conexões desse node. Porque fazemos isto ? Para controlar o número de processos criados, visto que não podemos ter um grande número de processos ativos simultaneamente. Esta foi uma das implementações que utilizamos para gerir processos.
\par O array \emph{nodes} guarda todos os nodes já criados, logo se quisermos conectar nodes que não existam, ou pelo menos um deles não exista, temos de avisar que tal é impossível e retornar um erro.
\par Já a \emph{matriz} serve para saber que nodes estão conectados, e atualizar essas conexões. Assim, após matarmos (caso exista) a conexão antiga entre o primeiro node (de onde vai sair a informação) e os restantes nodes (que vão receber a informação), vamos atualizar a matriz e de seguida através desta vamos criar um processo que ligue esse node aos nodes que já estava ligado e aos novos nodes.
\par Esta foi a função mais demorada a concluir, pois existia sempre erros minímos que nos adulteravam os resultados. Passámos algumas horas de volta desta, mas concluímo-la com sucesso.

\subsection{Disconnect}
\par A \emph{disconnect} (tal como percebemos pelo nome) faz o contrário da connect. Esta recebe uma lista de nodes e vai desconectar o primeiro dos restantes. Esta função foi rapidamente concluída, visto que a maneira que a implementámos é semelhante à maneira como implementámos a connect, simétrica, até, na prática. A única diferença é que em vez de colocar um \textbf{1} na matriz (significa que estão conectados), colocámos um \textbf{0} (significa que estão desconectados), tudo o resto é igual, ou seja, vai verificar se os nodes existem, se existem conexões.
\par Após destruir o processo vai criar os processos todos de novo que existiam e não tenham sido desconectados para aquele node. No entanto esta parte já estava implementada na connect também.

\subsection{Inject}
\par A inject recebe como argumentos um número, um cmd, e argumentos para o cmd. O número representa o node para o qual queremos mandar o output que o cmd dá ao fim de executar com os seus argumentos.
\par Por exemplo, \emph{./inject 1 ./const 10}, vai executar o componente \textbf{./const 10} e redirecionar o seu output para o node 1.
\begin{lstlisting}
pipe(nonamepipeout);
    p = fork();
    if(!p){
        close(nonamepipeout[0]);
        dup2(nonamepipeout[1], 1);
        close(nonamepipeout[1]);
        execv(argv[2], &argv[2]);
        execvp(argv[2], &argv[2]); //no caso do comando ser um pre definido o exec anterior falhara, executando entao um execvp
    }
    close(nonamepipeout[1]);
    int infd = open(id,	O_WRONLY);
    i = readln(nonamepipeout[0], buf, PIPE_BUF);
	write(infd, buf, i);
	fsync(infd);
\end{lstlisting}

\par Aqui podemos ver um excerto do código utilizado, onde vemos como implementamos, mais uma vez utilizamos um \emph{fork} para utilizar um \emph{exec} e usamos um \emph{dup2} para conseguir guardar todo o output gerado após a execução do exec. Após isto abrimos o \emph{fifo de entrada} do node id, e escrevemos todo o output lá.
\par Foi uma função feita sem grandes problemas, pois a base da implementação já tinha sido feito noutras funções do controlador.

\clearpage

\section{Caso de Teste}
\par Para melhor testar o nosso trabalho, criamos um caso de teste que, ainda que muito simplificado, tenta solucionar um problema real através de stream processing.
\par Imaginemos então que existe um software totalmente alocado numa \emph{cloud}, e que necessita de servidores para gerir os \emph{data sockets} que recebe e envia. Numa situação ideal em que o número de servidores é infinito, mas não existe maneira de o gerir, a seguinte rede poderá servir como um possível gestor de alocação de servidores, de maneira a aumentar a eficiência do sistema e a quantidade de servidores alocados num determinado momento ser ótima.

\begin{figure}[ht]
\centering
\includegraphics[scale=0.10]{test.jpg}
\caption{Rede de teste.}
\end{figure}

\par Num caso teórico em que existisse um comando \emph{alocar} em que alocava +1 ou -1 servidor dependendo do argumento, esta rede recebe os dados no formato id:numberofbytes, por exemplo, 7:4096. A rede guarda um entrylog.txt todos os sockets que recebe e filtra os dados em dois grupos, os que tem um número de bytes maior que a média dos últimos 10 e aqueles com menor. Guarda também os sockets que causaram uma mudança no número de servidores alocados. Assim, um possível ficheiro de configuração poderá ser:

\begin{verbatim}
node 1 const bytes
node 2 tee entrylog.txt
node 3 window 2 avg 2
node 4 filter 2 < 4
node 5 alocar -1
node 6 const -1
node 7 tee changelog.txt
node 8 const +1
node 9 filter 2 >= 4
node 10 alocar +1
connect 1 2
connect 2 3
connect 3 4 9
connect 4 5 6
connect 6 7
connect 8 7
connect 9 8 10
\end{verbatim}

\section{Conclusão}
\par O trabalho foi realizado com sucesso apesar de ter dado alguns problemas. Tivemos bastante dificuldade na parte de gerir processos e pipes, pois às vezes uns abriam e outros não, havia menos processos que o suposto, entre outros problemas. Tivemos algumas dor de cabeça até encontrar o porquê do problema e a sua solução.
\par Achamos, no entanto, que este trabalho nos ajudou bastante a perceber todas as funcionalidades aprendidas nesta \textbf{UC}, pois tivemos uma abordagem prática e por conta própria, que nos fez procurar soluções e métodos para desenvolver estas.
\par Houve aspetos e funcionalidades que não fomos capazes de concluir, como por exemplo a remoção de nodos ou a substituição das suas componentes, mas mesmo assim acreditamos que o resulado foi satisfatório e, acima de tudo, enriquecedor.
\par Após a conclusão deste trabalho, pensamos que todos os elementos deste grupo saem com uma percepção mais realista e clara da fragilidade dos processos quando não são bem geridos. Percebemos também que a maneira como abrimos e comunicamos entre os pipes tem uma enorme importância para termos o resultado esperado.
\end{document}
